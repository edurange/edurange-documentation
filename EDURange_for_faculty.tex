\documentclass[11pt]{report}
\usepackage{times}
\usepackage{fullpage}
\usepackage{url}
\usepackage{multirow}
\usepackage{rotating}
\usepackage{graphics}
\usepackage{graphicx}
\usepackage{amsmath, amssymb, graphicx}
\usepackage{amsmath, amssymb, graphicx, tikz, fancybox,relsize}

\newcommand{\eat}[1]{}
\renewcommand{\thesection}{\Alph{section}}

%from http://devdaily.com/blog/post/latex/control-line-spacing-in-itemize-enumerate-tags/
\newenvironment{packenum}{
\begin{enumerate}
  \setlength{\itemsep}{1pt}
  \setlength{\parskip}{0pt}
  \setlength{\parsep}{0pt}
}{\end{enumerate}}

\makeatletter
    \def\thebibliography#1{\chapter*{References Cited\@mkboth
      {REFERENCES CITED}{REFERENCES CITED}}\list
      {[\arabic{enumi}]}{\settowidth\labelwidth{[#1]}\leftmargin\labelwidth
        \advance\leftmargin\labelsep
        \usecounter{enumi}}
      \def\newblock{\hskip .11em plus .33em minus .07em}
      \sloppy\clubpenalty4000\widowpenalty4000
      \sfcode`\.=1000\relax}
    \makeatother

\begin{document}

\title{EDURange Instructor's Manual}
% Instructions for Faculty
% \author{Locasto, Weiss, Mache, Boesen}
\maketitle

% \input{intro} 
\section{Introduction}
\label{sec:intro}
{\bf  EDURange is working.  This document will be updated as changes are made.}
EDURange is both a collection of interactive, collaborative cybersecurity exercises and a framework 
for creating these exercises.  Currently, we have four exercises:  Recon, ELF Infection, strace and scapy hunt.
Recon was the first exercise created and was based on a scenario from PacketWars.  It
focuses on reconnaissance to determine hosts in an unknown network.  The standard tool for this is 
{\tt nmap}, and while the student will need to learn how to use that tool in order to do this exercise, 
that is
not the most important learning goal. The most important learning goal is developing the
analytical skills of understanding complex systems and complex data.  Similarly, the Elf Infection 
exercise uses standard tools such as {\tt netstat} but requires that students
reason about the behavior of a complex system to discover which binary is infected and 
what it is doing, e.g.
it opens a port and listens for connections, which it should not be doing.  
Scapy hunt is about listening passively to discover hosts on the local network and 
which other hosts they are talking to.
There are several more exercises planned, and they will be described as they become available.
An important part of these exercises is discussing them in class after the students have 
completed them

%we need an intro to EDURange/AWS lab -- who should write that?


\subsection{Using EDURange: AWS}
As an instructor, you will use an Instructor VM that will allow you to run scenarios
and gives you fine-grained control over starting and stopping instances.
The new URL for the instructor machine is \url{http://cloud.edurange.org:3000}
The EDURange team can set up your account and credentials.
You will also be able to monitor scoring events from the instructor machine.
Once you login, you will see three tabs at the top.  From the Home page,
you can create student groups.  These groups can be thought of as teams that will work
together. % (this has not been fully implemented yet).  
Click on 'scenarios' to get started.

From the browser, you can  start and stop scenarios on the instructor machine.
A scenario creates new VM instances and configures them.
There is an AWS console that also gives you a way
to start and stop any Amazon Instances (AMIs) that
were created by the scenarios on the instructor machine, but you shouldn't need to do that.
For each scenario, there a YAML file that specifies the exercise.
It includes a number of temporary student accounts and passwords for that scenario instance. 
Note that there are two different passwords for students.  There is a password 
for cloud.edurange.org and a temporary password for the student VM that is created by
the script.  It is the latter password that is set by the script.
Currently, you can change the IP addresses of the targets in Recon
by using the Change Answers text box in your browser, subject to the resource limitations 
of the account.  
This actually modifies the YAML file before running the scenario.
  In general, students will each have their 
own EC2 instances to log into and work on the exercises (first they connect through 
an external IP address to a Gateway). 
\eat{
The next section will lead you through 
starting an instructor machine and how to use it to create the scenarios.
There are two modes for using EDURange.  You may be using your own account or you may
be using the EDURange group account.  The use of those two modes will described separately.}

\subsubsection*{Using the instructor machine from the EDURange account}
If you are going to use the EDURange account, you will access the EDURange console
on \url{http://cloud.edurange.org:3000}
using the username and password that you receive from the administrators, 
and you will need to be a member of the edurange group or the edu\_fac group. 
There is a form on this website for you to request access, but you can also 
send e-mail to kaheah [at] gmail.com or weissr [at] evergreen.edu .  Once you have an account,
here are the basic steps to run a scenario:
\begin{itemize}
  \item Login to http://cloud.edurange.org:3000.  % screen shot
  \item Give registration code to students -- this allows them to register for their student account.
    Students then use the registration code to sign up.
  \item When students register and you refresh your home page, you will see that they are automatically 
    added to the group All.  You may create new groups and add students to those groups to simplify 
    administration.
  \item Now, you are ready to start a new scenario.  Click on the Scenarios button at the top.
    Then, you will see a page with buttons to create the different scenarios.
    Create the scenario you want, and you will see
    the students in the groups that you added to that scenario. This may take a few minutes.
    You can add and remove groups through this interface.
    When you have done that, you should click on Boot to start the VMs.  This will take about 5 min.
  \item To try the exercise as a student, you can use a separate terminal to ssh to the Gateway VM
    that was created by the script.  The Gateway will have an external IP address, visible
    from the AWS console, while the Team % does the script print this to the terminal?
    instances and Battlespace instances will not.
    You can assume the role of one of the players in the YAML file for the scenario, e.g.
    edurange\_1.  

\end{itemize}


% old \url{https://edurange.signin.aws.amazon.com/console}


  \eat{Describe the role of VPCs and VPNs in AWS.  What are the limits.
we need a video of this.
  At the upper right, there will be a dropdown tab
for the different AWS centers.  You want {\em  East (N Virginia)} and you should see a 
heading called {\bf Resources}.
Under that, click on the link to {\bf Running Instances}. You should see a list of instances.  You will
create your own instance of an instructor machine by taking a snapshot of an existing instructor instance,
 e.g. locasto-instructor.  }

 \eat{one of the scenarios and run a script that will create and configure the VMs required for that
scenario.  }

%\begin{enumerate}
\eat{
  \item Starting the instructor machine:
    Scroll down to Locasto-Instructor machine.  Look at its instance status.
    If it is ''running'' then it will have an external IP address and you can login to it.
    You can login to this VM with the credentials  ubuntu/edurange12.  
    If it is stopped, then click on the box next to the instance and go to the Actions
    pull-down tab at the top, and you can start it.  You can also right click on the 
    word ``stopped'' and you 
    should see an option to start it.  The circle in the status will turn from red to yellow and finally, to
    green. This can take 5 minutes.
 It will first say initializing, but eventually the status check will be complete and
    you can login.  On the lower part of the screen, you will see the public IP address,
    using ssh client (on Windows you can use PuTTY), 
    
     Be sure to stop the machine when you are done with
     the exercise.  {\bf Do not terminate it, since that will remove it completely}.
  \item 
    If you created your own Instructor machine, you must configure it.  You can set the username 
    and password, as well as a key pair, in the AWS console.  AWS can generate a key pair for you.
    There is additional information about Installation on the github:
    \url{http://www.github.com/sboesen/edurange/README/  }  % end eat
    % how long does it take to install s/w?

    Note: If you are new to EDURange, you can just use the existing Instructor machine and start it.
     \eat{The instances are on the US East cloud (N Virginia).  You can switch to 
    that in the drop down in the upper right of the window next to your name.  Then, you can
    click on running instances (there should be 0) under Resources.  }
  \item After logging in to the instructor machine, 
\begin{verbatim}
> cd edurange
\end{verbatim}
and run one of the scenarios:
\begin{verbatim}
> edurange scenarios/recon.yml or
> edurange scenarios/elf.yml 
\end{verbatim}
% how do you turn off debugging output?  The should be a success message with # instances started
\item the YAML script specifies the usernames and passwords for the player accounts.  
You can search the YAML script for ``Users'' or ``Password'' to find them.  Until we create a real UI,
you will need to distribute these to the appropriate students/players.  They use these credentials
to login directly to the Gateway VM and their Team Instance.  The Gateway has an external IP address,
and the students use the Gateway (NAT) rather than the console.
 Once logged in, they can 
change their passwords using standard commands on the VMs.
\item As instructor, you can see the instances being created and initialized in the AWS console view.  
  You can sort the instances by time launched.  You will not see the IP addresses unless
  they are public.  However, you can see the IP addresses in the YAML file if you search
  for ``IP\_Address''.  You can change these addresses on your own Instructor instance.  
}

%\end{enumerate}


\section{Exercises}

\begin{itemize}
\item {\bf Strace} (dynamic analysis of binaries) poses the challenge of 
 understanding what
a process is doing based on its system calls.  Students learn to filter large 
amounts of data to distinguish between normal and anomalous behavior.

\item {\bf Elf infection} assesses the student's understanding 
of the structure of an executable file.  The goal is to teach the student, having identified that a program is 
doing something malicious, where that code has been injected and how it works.  This 
is a reverse engineering problem and can use a range of tools, including readelf, objdump, gdb, 
strace and netstat.

\item {\bf Recon} is an exercise to examine how network protocols such as TCP, UDP, and ICMP 
can be used to reveal information about a network.  Recon 
focuses on reconnaissance to determine hosts in an unknown network. 
The student can explore tradeoffs between speed and stealth
when using tools such as nmap.
More advanced levels will address intrusion detection and network monitoring.

\item {\bf Fuzzing} is an attack-defense exercise where the defender must correctly implement a user 
interface that filters malformed expressions.  In the simplest version, the interface is for a 
calculator, and the defender must implement both the interface and the application.  
\eat{
\item {\bf Scapy Hunt} poses the challenge of analyzing network traffic 
to understand who is communicating with whom and how. The player is trying to get data from an ftp server
which is not on the same subnet, but one of the hosts on its network is communicating with it.
By default the player can only see packets sent to the server and must craft packets to get them routed
to the target and get a response back.

\item {\bf Firewall} is still being implemented.  It requires students to understand the 
interactions between multiple firewalls which have different rule sets.
} % end comment
\end{itemize}

\input{strace/strace_tutorial}


\eat{
\subsection{strace}
{\tt strace} is a useful tool for understanding how programs interact with the operating system through system calls.
%% what are some of the more useful options?  e.g. for looking at subprocesses

Prerequisite knowledge:
\begin{itemize}
   \item using ssh
   \item Linux command line: navigate using cd, ls, a text editor, such as nano, vi, or emacs, gcc
   \item Linux PATH conventions, echo \$PATH, the order of search for executables
   \item What a system call is
   \item A basic understanding of processes and subprocesses
\end{itemize}


Here are the associated learning goals:
\begin{itemize}
  \item know the capabilites of {\tt strace} and how to use the basic options. 
  \item know which system calls do the following: 
    \begin{itemize} 
      \item make a new name for a file, 
      \item execute a process, 
      \item terminate the calling process
      \item create message buffer and read from message queue
      \item assign the local IP address and port for a socket
    \end{itemize}
  \item know the general classes of system calls.
  \item Be able to read a system trace and know what is normal vs abnormal.
  \item be able to make a system call in C
  \item be able to make a system call in x86 assembly
  \item understand how the kernel handles system calls
  \item understand how some system calls introduce threats
  \item understand how errors are handled
\end{itemize}
}  %end eat



\subsection{Elf Infection}
The Elf infection exercise only uses a single VM for each team.  

Prerequisite knowledge:
\begin{itemize}
   \item Linux command line: navigate using cd, ls, a text editor, such as nano, vi, or emacs, gcc
   \item How to run gdb, objdump
   \item strace is helpful
   \item How x86 assembly works, including branches.
   \item A basic understanding of the segments and sections of an ELF file.
\end{itemize}

Here are the associated learning goals:
\begin{itemize}
  \item know the capabilites of {\tt readelf} and how to use the basic options.
  \item know the format for the header of an ELF file.
  \item know which system calls do the following: 
    \begin{itemize} 
      \item make a new name for a file, 
      \item execute a process, 
      \item terminate the calling process
      \item create message buffer and read from message queue
      \item assign the local IP address and port for a socket
    \end{itemize}
  \item know the general classes of system calls.
  \item Be able to read a system trace and know what is normal vs abnormal.
  \item be able to make a system call in C
  \item be able to make a system call in x86 assembly
  \item understand how the kernel handles system calls
  \item understand how some system calls introduce threats
  \item understand how errors are handled
\end{itemize}



\subsection{Recon}
\subsubsection{Learning Outcomes for Recon}
%The learning goals for the reconnaissance exercise are:
Students will answer the following questions:
\begin{enumerate}
   \item what is the 3-way handshake?
   \item what does the SYN flag do?
   \item What does /17 mean?  how many IP addresses does that include?
   \item What are the options for nmap and what are their differences in terms of time, stealth and
   protocols?
   \item which ports does nmap -sT scan?
   \item which ports does nmap -sU scan?
\end{enumerate}

\begin{itemize}
\item understanding networking protocols (TCP, UDP, ICMP) and how they can be exploited for recon.
\item developing the security mindset
\item understanding CIDR network configuration and how subdivide a network IP range.
\item use nmap to find hosts and open ports on a network.
\end{itemize}

{\bf Diagram of Recon network}
% use tikz to draw the network configurations
\usetikzlibrary{arrows,decorations.pathmorphing,backgrounds,positioning,fit,petri}
\tikzstyle{tiny} = [inner sep=0pt, minimum size=2pt, text centered, fill=blue!20]
\tikzstyle{medium} = [inner sep=4pt, minimum size=4pt, text centered, fill=blue!20]
\tikzstyle{subnet} = [fill=blue!20]
\tikzstyle{rectangle} = [align=center]

\begin{figure}[h]
\hfill
\begin{center}

\begin{tikzpicture}[node distance=10mm, >=latex]
%\draw (-1.5,-1) circle (0.5);
   \node (Internet) at (0, 1) [rectangle, draw] {$Internet$};
   \node (Student) at (4,0) [rectangle, draw] {$Student$} edge[<->] (Internet);
   \node (Instructor) at (-4, 0) [rectangle, draw] {$Instructor$} edge[<->] (Internet);
   \node (NatSubnet) at (0, -2) [rectangle, subnet, draw] {NAT \\ (10.0.128.0/28)} edge[<->] (Internet);
   \node (NatInstance) at (6, -2) [rectangle, draw] {Nat Instance \\ (10.0.128.4)} edge[<->] (NatSubnet);

   \node (PlayerTwoSubnet) at (6, -4) [rectangle, subnet, draw] {Player 2 \\ (10.0.128.32/28)} edge[<->] (NatSubnet);
   \node (PlayerTwoInstance) at (6, -6) [rectangle, draw] {Player 2 Instance \\ (10.0.128.34)} edge[<->] (PlayerTwoSubnet);

   \node (PlayerOneSubnet) at (-6, -4) [rectangle, subnet, draw] {Player 1 \\ $(10.0.128.16/28)$} edge[<->] (NatSubnet);
   \node (PlayerOneInstance) at (-6, -6) [rectangle, draw] {Player 1 Instance \\ (10.0.128.18)} edge[<->] (PlayerOneSubnet);

   \node (BattleSpace) at (0, -4) [rectangle, subnet, draw] {Battle Space \\ (10.0.127.255/17)} edge[<->] (NatSubnet);
   \node (BattleSpaceInstance) at (0, -6) [rectangle, draw] {BattleSpace Instance 1 \\ (10.0.3.3)} edge[<->] (BattleSpace);

\end{tikzpicture}
\caption{Conceptual diagram of the Recon I game. Note subnets are blue, and IP addresses are just
                    as an example. The Battlespace is currently 10.0.0.0/17, but you can 
                    make your own copy of the scenario file and change this.  
                    You can also change the IP addresses of the targets.  
}
\end{center}
\end{figure}


\subsection{Fuzzing}
The Fuzzing scenario runs with 3 primary instances: scoring, defending and attacking, plus a NAT.

To test this scenario - first log into the defending instance by first logging into the NAT and 
then logging into the defending team's machine (defending\_team@10.0.26.130), and make a calculator, 
the author of this scenario has provided a na\"{i}ve calculator downloadable here 
(\url{http://ada.evergreen.edu/~weidav02/crashing_calc.c}). After compiling your calculator, 
you may submit it by running the following command (submit\_calc /path/to/calc).

After submitting your calculator, you are ready to fuzz. Connect to the attacking team's machine 
(attacking\_team@10.0.26.129), You may run my naive fuzzer example downloadable here 
(\url{http://ada.evergreen.edu/~weidav02/example_fuzz.py}) or you may try testing the calculator by hand 
with examples such as these following commands: \\
(\url{send_fuzz_data "22"}), or \\
(\url{send_fuzz_data "$(perl -e 'print "\x03"')"}).   \eat{$}
Sending a hexidecimal character such as 0x03 is a 
good test to run as it is outside the valid character set for this calculator and should cause errors.


To view the scoreboard, you must log into the scoring machine (\url{scorer@10.0.26.128}) 
and run this command (cat /tmp/scoring/answers) - 
This will be updated in the next few days so that players may view the scoreboard from their machines.


%\subsection{Scapy Hunt}

\section{Future Work}
\begin{itemize}
 \item ** in progress **  Make it possible for a student to take a snapshot and stop and restart the
   exercise.
 \item We plan to create an API to choose random IPs for the Battlespace and other randomizations (passwords).
 \item Scorebot.  A simple version is running.
 \item Instructor versions: each instructor can have his/her own YAML scripts that will be used to 
   generate the scenarios page.
\end{itemize}


\section{References}


\section{Student Tutorial}
\subsection{ What are TCP and UDP?  }
In order to understand this exercise, you should be familiar with the 3-way handshake for TCP.
You should also know something about ICMP and UDP.
You will learn in this exercise, how these and other protocols can be used to discover hosts on a network,
which ports on those hosts are open, and what applications are running on them.
In practice, each message that is sent over the Internet 
uses multiple protocols, which are divided into five layers: physical layer, link layer, network layer,
transport layer and application layer.
For example, the physical layer handles what is a 0 or 1.
The link layer handles communication on local area networks (LANs).
The network layer handles rounting on wide area networks (WANs), e.g. IP.
The transport layer handles ports and processes, e.g. TCP, UDP, ICMP.
The application layer handles applications communicating with each other, e.g. http.
by nesting packets inside of packets.  In general,
these packets correspond to layers  of functionality:   
TCP is connection-oriented and is responsible for 
a number of things including reliably conveying messages between the application layers on two hosts.  
The three-way handshake establishes this pairing with the following sequence: SYN, SYN-ACK,  and ACK
You can get a summary of the important protocols and their layers in:
Chapter 4 of Hacking (Erickson)[1] or Chapter 2 of Counter Hack Reloaded [2].  
Network Security by Kaufman, Perlman, Speciner

subnets: how to split an IP range, IP CIDR notation

tcpdump (man tcpdump) on VM
Wireshark will sniff network traffic.  Although it may not capture 
all traffic on your network, it will at least show you what you network interface can see,
including all of the messages that your computer is sending.  
This can be very useful when you are figuring out what nmap is doing.  Since we are not
currently using Xwindow forwarding, 


Linux command line: man man, cd, ls , pwd, mkdir, 
teach the basics of Linux, find the file called password.txt

\subsection{nmap}
 How does nmap work?
nmap generates packets, sends them, and examines the responses (if there are any).
For example, nmap -sT 192.168.12.1  will send TCP packets to ports 1-1024 at IP address 192.168.12.1.
In order to understand what nmap is doing, we will use tshark, which is a CLI for Wireshark.
read [3] on Wireshark and t-shark.  The parameters to check on nmap include 
-sS, -sP, -sT, -sU, what about the -T option and -n?

This is important so students can see what is happening.
Can you see the 3-way handshake?  Is the IP address of the attack host visible?
What happens if you spoof the return IP address?  Can nmap do that?
What does nmap -sS do?
Suppose you want to map the network faster, without being stealthy?  What are the -T options?
How would you detect that a computer is scanning your network, running t-shark?

Tips: use -n for no DNS lookup.  What is DNS? read [4]
nmap has lots of options, so focus on the following first: -T, -sS, -sP, -sT, -n, -o


Using EDURange (students): 
In order to get started, you need to understand to connect to Amazon's AWS, the gateway and team hosts.
You need some kind of credentials to login.  You need to login to the a gateway on Amazon in order to use 
EDURange, but first you will need a password.  You will be given a URL at the time of the exercise that will 
tell you your password and that will be the gateway computer.
Note that for now, pw will be generated randomly by the yaml parser.
navigate to gateway url in browser and get a browser-based ssh client,  which will ask for uname/password

ideas for level-0 edurange games: debug challenges, clone from github.  {\tt netcat} on port 8000, why doesn't work? 
ARP tabl, delete gateway,






1 Hacking: The Art of Exploitation(Chapter..)

2 Counterhack Reloaded (Chapter 2)

3 man pages for nmap


\section*{contributors}
Stefan Boesen, Erin Davis, Michael Locasto, Jens Mache,  Lyn Turbak, Richard Weiss, 

%\bibliographystyle{plain}
%\bibliography{livefire}

\end{document}
